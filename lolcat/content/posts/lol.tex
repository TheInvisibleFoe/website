\documentclass[a4paper]{article}

\usepackage[utf8]{inputenc}
\usepackage[T1]{fontenc}
\usepackage{textcomp}
\usepackage[dutch]{babel}
\usepackage{amsmath, amssymb}
% \usepackage{preamble}
\usepackage{mlmodern}
\usepackage{transparent}
\newcommand{\incfig}[1]{%
    \def\svgwidth{\columnwidth}
    \import{./figures/}{#1.pdf_tex}
}
\pdfsuppresswarningpagegroup=1
\title{The Magnus Effect}
\begin{document}
\maketitle
    
\section{Magnus effect }
This article aims to further strengthen the understanding of the Magnus effect by finding a direct correlation between the Bernoulli Fluid Equation and the Magnus force. The Magnus force is force that affects "rapidly" spinning bodies in the direction perpendicular to their movement. Now here comes the construction from the Bernoulli equation.
\subsection{Analytical approach}
Suppose a cylinder moves forward so the "faster" moving surface "pulls" the air up and in turn gets pulled down due to newton's Third law.
\subsection{Scenario}
There is a cylinder moving in space in the direction $+\hat{x}$. Now the cylinder is given an angular velocity $\omega$ which points in the $+\hat{k}$ direction.Therefore the ball spins counter clockwise.
\subsection{Derivation}
We start of with the Bernoulli equation from fluid dynamics.
In this case the terms are as follows: $P_0$ refers to the ambient pressure, $\frac{1}{2}\rho v^2$ refers to the dynamic pressure, the last term can be ignored here as the cylinder used for measuring the Magnus effect can be taken to be arbitrarily small causing the height difference to be negligible.
\begin{equation}
    P_0 + \frac{1}{2}\rho v^2 +\rho_0 gh = \kappa \\
\end{equation}
The ambient pressure change essentially gives rise to the change in the pressure between the high and low spinning sides of the cylinder causing the Magnus effect. So here the term $v$ represents the fluid velocity. here the fluid velocity is the velocity is the velocity of a small elemental part of the fluid. Now let us define two pressures of the cylinder in question. Let the top pressure be denoted by $P_t$ and the bottom surface pressure be denoted by $P_b$. Now let the ball move from left to right in this question (suppose$\hat{x}$ with velocity u).
\begin{align*}
    P_t &= \kappa - \frac{1}{2}\rho(v-r\omega)^2\\
    P_b &= \kappa - \frac{1}{2}\rho(v+r\omega)^2\\
\end{align*}
To understand the above equations let us enter the COM frame of the object(only translation frame not the spinning frame). This essentially allows us to set the velocity of air in direction $-\hat{x}$. So now we need the fluid velocity at the top of the cylinder and the bottom of the cylinder. The fluid velocity is the relative velocity of the air with respect to the top surface of the cylinder. At the top part of the cylinder we have $v_r = v-r\omega$ as the ball is spinning in direction opposite to the velocity at that point. Similarly the expression for the bottom velocity can also be defined.
\begin{align*}
    \Delta P &= |-\frac{1}{2}\rho(v-r\omega)^2 +\frac{1}{2}\rho(v+r\omega)^2|\\
    \Delta P &= |2\rho v r \omega|
\end{align*}
This $\Delta P$ is in the net direction downward i.e. in the direction $\omega \times v$. This net pressure change gives us an important fact that there will be a net force downward causing the ball to 'curl'.
\subsection{Conclusion}
So essentially we can define the Magnus force as:
\begin{align*}
\fbox{$\vb{F_{magnus}} = S(v)\vb*{\omega\times v}$}
\end{align*}
Now there is a lack of published formula on the Magnus Effect because there are multiple formulae that yield different results. We can assume irrotational flow which gives us the above definition of the Magnus effect. But , without assuming the irrotationality of the flow we can assume that the liquid is being pulled away due to the centrifugal force and thus needs an added pressure gradient to balance out the centrifugal force experienced by the liquid. Further on that later.
\subsection{Post Notes}
The defined formula for the Magnus effect has not been published yet as the defined formulae come out to be different for different models of the fluid flow taken and the assumptions made in the liquid's movement. The general formula written above is True but the formula for $S(v)$ is not defined properly.
\subsection{Extra Reading}
\begin{enumerate}
    \item Irrotational Flow in Fluids(Richard Fitzpatrick)
    \item Potential Flow around a cylinder
    \item David Tong (Fluid Dynamics)
    \item Flow past a circular cylinder
\end{enumerate}
\end{document}
